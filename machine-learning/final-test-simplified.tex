\documentclass[a4paper,10pt]{article}
\usepackage[T2A]{fontenc}
\usepackage[english,russian]{babel}
\usepackage[utf8]{inputenc} 
\usepackage[vmargin={1cm,2cm},hmargin={2cm,2cm}]{geometry}
\usepackage{graphicx}

\usepackage{amssymb}
\usepackage{amstext}
\usepackage{amsmath}
\usepackage[warn]{mathtext}

\usepackage{indentfirst}
\usepackage{wrapfig}
\usepackage{topcapt}

\title{Финальный зачёт по машинному обучению}
\newcounter{LFMSHnumber}
\setcounter{LFMSHnumber}{\the \year}
\addtocounter{LFMSHnumber}{-1987}
\date{ЛФМШ-\arabic{LFMSHnumber}, август {\the \year}~г.}

\newcommand{\figRef}[1]{рис.~\ref{#1}}
\newcommand{\FigRef}[1]{Рис.~\ref{#1}}
\newcommand{\Example}{\textbf{Пример: }\par}

\graphicspath{{graph/}}

\begin{document}
\maketitle
\begin{enumerate}
	\item Снова напишите свою фамилию, имя и отряд разборчивым почерком.
	\vspace{1cm}
	\item Для каких из перечисленных алгоритмов характерна проблема переобучения?
	\begin{enumerate}
		\item Решающее дерево
		\item Метод kNN
		\item Логистическая регрессия
		\item Характерна для всех
	\end{enumerate}
	\item Каким образом устроена функция ошибки в методе логистической регрессии?
	\begin{enumerate}
		\item Минимизируется величина $Accuracy = \frac{TP+TN}{TP+TN+FP+FN}$ на обучающей выборке
		\item Минимизируется несоответствие предсказания с разметкой $L=\sum |P(\vec x^i) - c_i|$
		\item 
		\begin{sloppypar}
			Максимизируется вероятность правильной классификации обучающей выборки ${P_{train}=\prod_{i=1}^N (P(\vec x^i))^{y_i} (1 - P(\vec x^i))^{1 - y_i}}$
		\end{sloppypar}
		\item Минимизируется логистическая функция  $f(x)=\frac{1}{1+e^{-x}}$ от расстояния до гиперплоскости $d(\vec r)$
	\end{enumerate}

	
	\item
	Напишите уравнение плоскости в $n$-мерном пространстве, если заданы единичная нормаль $\vec n$ и некоторая произвольная точка $\vec r_0$, лежащая на плоскости
	\vspace{3cm}
	
	\item
	Объясните, что делают следующие строчки кода:
	\begin{verbatim}
		clf = LogisticRegression()
		clf.fit(X_train, y_train)
		y_predict = clf.predict(X_val)
	\end{verbatim}
	\vspace{5 cm}
	\item В чём состоит метод логистической регрессии? Почему его называют <<регрессия>> (предсказание вещественной величины), а не <<классификация>> (предсказание бинарного класса)? (Можно перевернуть листочек и писать там!)
	\vspace{4cm}

\end{enumerate}
\end{document}

