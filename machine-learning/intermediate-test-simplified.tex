\documentclass[a4paper,10pt]{article}
\usepackage[T2A]{fontenc}
\usepackage[english,russian]{babel}
\usepackage[utf8]{inputenc} 
\usepackage[vmargin={3cm,2cm},hmargin={2cm,2cm}]{geometry}
\usepackage{graphicx}

\usepackage{amssymb}
\usepackage{amstext}
\usepackage{amsmath}
\usepackage[warn]{mathtext}

\usepackage{indentfirst}
\usepackage{wrapfig}
\usepackage{topcapt}
\usepackage{multirow}

\title{Промежуточный зачёт по машинному обучению}
\newcounter{LFMSHnumber}
\setcounter{LFMSHnumber}{\the \year}
\addtocounter{LFMSHnumber}{-1987}
\date{ЛФМШ-\arabic{LFMSHnumber}, август {\the \year}~г.}


\newcommand{\figRef}[1]{рис.~\ref{#1}}
\newcommand{\FigRef}[1]{Рис.~\ref{#1}}
\newcommand{\Example}{\textbf{Пример: }\par}

\graphicspath{{graph/}}


\begin{document}
\maketitle
\begin{enumerate}
	\item Напишите свою фамилию, имя и отряд разборчивым почерком
	\vspace{1cm}
	\item Что такое переобучение?
	\begin{enumerate}
		\item Невозможность минимизации функции ошибки
		\item Чрезмерная адаптация модели к особенностям обучающей выборки с потерей обобщающей способности
		\item Чрезмерное уширение решающего дерева
		\item Занятия т.\,н. <<спортивным программированием>> более 8\,ч. в день
	\end{enumerate}
	
	\item Расставьте TP, FP, TN, FN по матрице ошибок (confusion matrix) для бинарного классификатора:
	\begin{center}
		\begin{tabular}{ |c|c|c|c| } 
			\hline
			 \multicolumn{2}{|c|}{} & \multicolumn{2}{|c|}{Истинные значения}  \\
			 \cline{3-4} 
			 \multicolumn{2}{|c|}{} & True & False \\ 		 
			 \hline
		     \multirow{2}{5em}{Результаты модели} & True &  &  \\ 
		     \cline{2-4}
			 
			& False &  &  \\ 
			\hline
		\end{tabular}
	\end{center}	

	\item Опишите, в чём заключается нормализация данных и для чего она нужна:

	\newpage
	\item Кратко опишите суть метода kNN.
	\vspace{10cm}
	\item Чем задача кластеризации отличается от задачи классификации?
	\vspace{9cm}
	
\end{enumerate}
\end{document}

