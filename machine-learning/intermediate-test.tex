\documentclass[a4paper,10pt]{article}
\usepackage[T2A]{fontenc}
\usepackage[english,russian]{babel}
\usepackage[utf8]{inputenc} 
\usepackage[vmargin={3cm,2cm},hmargin={2cm,2cm}]{geometry}
\usepackage{graphicx}

\usepackage{amssymb}
\usepackage{amstext}
\usepackage{amsmath}
\usepackage[warn]{mathtext}

\usepackage{indentfirst}
\usepackage{wrapfig}
\usepackage{topcapt}

\title{Зачёт по машинному обучению}
\newcounter{LFMSHnumber}
\setcounter{LFMSHnumber}{\the \year}
\addtocounter{LFMSHnumber}{-1987}
\date{ЛФМШ-\arabic{LFMSHnumber}, август {\the \year}~г.}


\newcommand{\figRef}[1]{рис.~\ref{#1}}
\newcommand{\FigRef}[1]{Рис.~\ref{#1}}
\newcommand{\Example}{\textbf{Пример: }\par}

\graphicspath{{graph/}}


\begin{document}
\maketitle
\begin{enumerate}
	\item Что такое переобучение?
	\begin{enumerate}
		\item Отклонение градиентов от направления возрастания функции ошибки
		\item Чрезмерная адаптация модели к особенностям обучающей выборки с потерей обобщающей способности
		\item Слияние большого числа мелких кластеров в один большой
		\item Занятия спортивным программированием более 8\,ч. в день
	\end{enumerate}

	\item Отметьте, какие из этих алгоритмов и задач требуют нормализации данных для эффективной работы:
	\begin{enumerate}
		\item Решающее дерево
		\item Метод kNN (k ближайших соседей)
		\item Алгоритмы кластеризации, в которых расстояние измеряется по ближайшим элементам кластеров
		\item Алгоритмы кластеризации, в которых расстояние измеряется между центрами кластеров
	\end{enumerate}
	
	\item Что такое функция ошибки (loss-функция)?
	\begin{enumerate}
		\item Функция распределения ошибки классификации по датасету
		\item Функция, характеризующая потерю точности предсказаний модели при переходе от обучающей выборки к валидационной
		\item Функция, характеризующая отличие предсказаний модели от разметки
		\item Функция, выполняющая предсказывающая случайным образом. Используется для сравнения предсказаний модели со случайным угадыванием
		\item Традиционный ритуальный головной убор северо-африканских племён в 16 веке. Использовался при совершении обряда жертвоприношения для избежания ошибок.
	\end{enumerate} 
		
	\item Кратко опишите, в чём суть метода kNN.
	
	\item Чем задача кластеризации отличается от задачи классификации?
	
	% \item Предложите, как использовать метод kNN в случае, когда среди признаков есть как вещественные числа, так бинарные (1 или 0). Пример: у наблюдаемых больных измеряют температуру, артериальное давление (вещественные числа), а также наличие головных болей (есть или нет).
	
	\item Энтомологи изучают неизвестные ранее виды жуков на далёких островах. Они ловят насекомых, взвешивают их в граммах, измеряют длину их тельца, головы и конечностей в миллиметрах, записывают количество пятнышек (натуральное число), оценивают контрастность окраса по субъективной шкале от 1 до 5. Как машинное обучение может помочь оценить количество видов жуков, которые обитают на далёких островах? На что нужно обратить внимание? Предложите алгоритм действий.
	
	%\item Попробуйте вспомнить формулы для precision, recall, accuracy через TP (true positive), TN (true negative), FP (false positive), FN (false negative), и написать их.
\end{enumerate}
\end{document}

