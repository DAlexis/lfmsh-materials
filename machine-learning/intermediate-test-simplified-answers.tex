\documentclass[a4paper,10pt]{article}
\usepackage[T2A]{fontenc}
\usepackage[english,russian]{babel}
\usepackage[utf8]{inputenc} 
\usepackage[vmargin={3cm,2cm},hmargin={2cm,2cm}]{geometry}
\usepackage{graphicx}
\usepackage{color,soulutf8}

\usepackage{amssymb}
\usepackage{amstext}
\usepackage{amsmath}
\usepackage[warn]{mathtext}

\usepackage{indentfirst}
\usepackage{wrapfig}
\usepackage{topcapt}
\usepackage{multirow}

\title{Промежуточный зачёт по машинному обучению}
\newcounter{LFMSHnumber}
\setcounter{LFMSHnumber}{\the \year}
\addtocounter{LFMSHnumber}{-1987}
\date{ЛФМШ-\arabic{LFMSHnumber}, август {\the \year}~г.}


\newcommand{\figRef}[1]{рис.~\ref{#1}}
\newcommand{\FigRef}[1]{Рис.~\ref{#1}}
\newcommand{\Example}{\textbf{Пример: }\par}

\graphicspath{{graph/}}


\begin{document}
\maketitle
\begin{enumerate}
	\item Напишите свою фамилию, имя и отряд разборчивым почерком~[1\,бл]
	
	Дубинин Ваня Петрович
	
	\item Что такое переобучение?~[1\,бл]
	\begin{enumerate}
		\item Невозможность минимизации функции ошибки
		\item \hl{Чрезмерная адаптация модели к особенностям обучающей выборки с потерей обобщающей способности}
		\item Чрезмерное уширение решающего дерева
		\item Занятия т.\,н. <<спортивным программированием>> более 8\,ч. в день
	\end{enumerate}
	
	\item Расставьте TP, FP, TN, FN по матрице ошибок (confusion matrix) для бинарного классификатора:~[1\,бл]
	\begin{center}
		\begin{tabular}{ |c|c|c|c| } 
			\hline
			 \multicolumn{2}{|c|}{} & \multicolumn{2}{|c|}{Истинные значения}  \\
			 \cline{3-4} 
			 \multicolumn{2}{|c|}{} & True & False \\ 		 
			 \hline
		     \multirow{2}{5em}{Результаты модели} & True & \hl{TP} & \hl{FP} \\ 
		     \cline{2-4}
			 
			& False & \hl{FN} & \hl{TN} \\ 
			\hline
		\end{tabular}
	\end{center}	
	\item Опишите, в чём заключается нормализация данных и для чего она нужна:~[2\,бл]
	
	Номализация данных нужна для того, чтобы привести все признаки к одному диапазону. Чтобы нормализовать признак $x$ нужно:
	\begin{itemize}
		\item Найти среднее значение: $\overline{x} = \frac{\sum_{i=1}^N x_i}{N}$
		\item Найти среднеквадратичное отклонение: $\sigma_x = \sqrt{\frac{\sum_{i=1}^N (x_i - \overline{x})^2}{N}}$
		\item Преобразовать значения признака $x$ для каждого элемента выборки: $x^*_i = \frac{x_i - \overline{x_i}}{\sigma_x}$
	\end{itemize}
	
	Без нормализации невозможно корректно определять расстояния между точками в пространстве признаков

	\item Кратко опишите суть метода kNN.~[2\,бл]
	
	Для пробной точки, которую нужно классифицировать, находятся $k$~ближайших точек обучающей выборки, где $k$~"--- нечётное. Точек какого класса среди них больше, к такому классу и относят пробную точку. Для метода kNN необходима предварительная нормализация данных, потому что нужно считать расстояния.
	
	\item Чем задача кластеризации отличается от задачи классификации?~[2\,бл]
	
	Задача классификации~"--- это пример обучения с учителем, где модель должна по обучающей выборке, для которой известны <<ответы>>, научиться классифицировать незнакомые точки.
	
	Задача кластеризации~"--- это пример обучения без учителя, когда никаких классов заранее неизвестно, и алгоритм должен сгруппировать данные в кластеры на основании близости точек в пространстве параметров тем или иным способом.
	
\end{enumerate}
\end{document}

