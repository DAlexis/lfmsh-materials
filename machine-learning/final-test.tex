\documentclass[a4paper,10pt]{article}
\usepackage[T2A]{fontenc}
\usepackage[english,russian]{babel}
\usepackage[utf8]{inputenc} 
\usepackage[vmargin={1cm,2cm},hmargin={2cm,2cm}]{geometry}
\usepackage{graphicx}

\usepackage{amssymb}
\usepackage{amstext}
\usepackage{amsmath}
\usepackage[warn]{mathtext}

\usepackage{indentfirst}
\usepackage{wrapfig}
\usepackage{topcapt}

\title{Финальный зачёт по машинному обучению}
\newcounter{LFMSHnumber}
\setcounter{LFMSHnumber}{\the \year}
\addtocounter{LFMSHnumber}{-1987}
\date{ЛФМШ-\arabic{LFMSHnumber}, август {\the \year}~г.}

\newcommand{\figRef}[1]{рис.~\ref{#1}}
\newcommand{\FigRef}[1]{Рис.~\ref{#1}}
\newcommand{\Example}{\textbf{Пример: }\par}

\graphicspath{{graph/}}

\begin{document}
\maketitle
\begin{enumerate}
	\item Снова напишите свою фамилию, имя и отряд разборчивым почерком.
	\vspace{1cm}
	\item Для каких из перечисленных алгоритмов характерна проблема переобучения?
	\begin{enumerate}
		\item Решающее дерево
		\item Метод kNN
		\item Логистическая регрессия
		\item Перцептрон
		\item Нейронная сеть
	\end{enumerate}
	\item Каким образом устроена функция ошибки в методе логистической регрессии?
	\begin{enumerate}
		\item Минимизируется несоответствие предсказания с разметкой с квадратами $L=\sum (P(\vec x^i) - c_i)^2$
		\item Минимизируется несоответствие предсказания с разметкой с модулями $L=\sum |P(\vec x^i) - c_i|$
		\item 
		\begin{sloppypar}
			Максимизируется вероятность правильной классификации обучающей выборки ${P_{всё верно}=\prod_{i=1}^N (P(\vec x^i))^{c_i} (1 - P(\vec x^i))^{1 - c_i}}$
		\end{sloppypar}
		\item Вычисляется свёртка от выходов первого слоя с последующим применением логистической функции $f(x)=\frac{1}{1+e^{-x}}$
	\end{enumerate}
	
	\item Что из перечисленного отличает свёрточную нейронную сеть от полносвязной? Выберите все верные варианты
	\begin{enumerate}
		\item Свёрточная сеть использует функцию активации ReLU, а полносвязная~"--- логистическую функцию 
		\item Свёрточная сеть учитывает взаимное расположение пикселов, а полносвязная~"--- нет
		\item Свёточная нейронная сеть имеет значительно меньшее число настраиваемых параметров
		\item Свёточная нейронная сеть имеет значительно большее число настраиваемых параметров, и поэтому более эффективна
	\end{enumerate}
	
	\item
	Напишите уравнение плоскости в $n$-мерном пространстве, если заданы единичная нормаль $\vec n$ и расстояние до начала координат d.
	\vspace{1cm}
	\item Нарисуйте схему перцептрона.
	\vspace{4cm}
	\item В чём состоит задача понижения размерности? Можно продолжать писать на другой стороне листочка!
	\vspace{4cm}

\end{enumerate}
\end{document}

