\documentclass[a4paper,10pt]{article}
\usepackage[T2A]{fontenc}
\usepackage[english,russian]{babel}
\usepackage[utf8]{inputenc} 
\usepackage[vmargin={3cm,2cm},hmargin={2cm,2cm}]{geometry}
\usepackage{graphicx}

\usepackage{amssymb}
\usepackage{amstext}
\usepackage{amsmath}
\usepackage[warn]{mathtext}

\usepackage{indentfirst}
\usepackage{wrapfig}
\usepackage{topcapt}

\title{Список лабораторных работ}
\newcounter{LFMSHnumber}
\setcounter{LFMSHnumber}{\the \year}
\addtocounter{LFMSHnumber}{-1987}
\date{ЛФМШ-\arabic{LFMSHnumber}, август {\the \year}~г.}


\newcommand{\figRef}[1]{рис.~\ref{#1}}
\newcommand{\FigRef}[1]{Рис.~\ref{#1}}
\newcommand{\TwoLines}[2]{\begin{array}{@{}c@{}} #1 \\#2 \end{array} }
\newcommand{\Example}{\textbf{Пример: }\par}

\graphicspath{{graph/}}

\DeclareMathOperator{\Tr}{Tr}

\newcommand{\labtitle}[5]{
	\textbf{#2}\par
	\textit{#1 класс}\par
	\textbf{Цель работы:} #3\par
	\textbf{Оборудование:} #4\par
	\textbf{Тема:} #5
}

\begin{document}
\maketitle
\begin{enumerate}
	\item \labtitle
		{9}
		{Определение массы твёрдого тела}
		{измерить массу продолговатого тела с помощью имеющегося оборудования.}
		{указка, гирька известной массы, линейка, нитка, стол.}
		{центр масс. Правило рычага.}
	\item \labtitle
		{9}
		{Определение массы плоской фигуры}
		{измерить массу плоской фигуры, не используя весы.}
		{плоское тело, линейка, гирька известной массы, стол.}
		{центр масс, правило рычага}
	\item \labtitle
		{9}
		{Определение плотности пластилина}
		{определить плотность пластилина.}
		{пластилин, линейка, цилиндрический сосуд с водой.}
		{Закон Архимеда, условие плавания тел}
	\item \labtitle
		{9}
		{Определение плотности неизвестного металла}
		{найти плотность неизвестного металла и по таблице плотностей определить, какой металл был дан.}
		{цилиндрический сосуд с водой, пластиковый стаканчик, мелкие кусочки металла, линейка.}
		{Закон Архимеда, условие плавания тел}
	\item \labtitle
		{9}
		{Определение плотности дерева}
		{определить плотность дерева.}
		{узкий цилиндрический сосуд с водой, линейка, множество деревянных палочек естественной формы.}
		{Закон Архимеда, условие плавания тел}
	\item \labtitle
		{9}
		{Определение плотности неизвестной жидкости}
		{определить плотность неизвестной жидкости.}
		{цилиндрический сосуд, вода, неизвестная жидкость, линейка, пенопласт, пластилин.}
		{Закон Архимеда, условие плавания тел}
	\item \labtitle
		{9}
		{Определение массы тяжелого (или легкого) груза}
		{определить массу груза, исключающего непосредственное измерение.}
		{тяжелый груз, верёвка, линейка, динамометр, не позволяющий измерить массу груза непосредственным взвешиванием.}
		{второй закон Ньютона, разложение сил на составляющие}
	\item \labtitle
		{9}
		{Исследование неидеального источника напряжения}
		{получить зависимость тока в цепи и напряжения на нагрузке от её сопротивления. Постараться объяснить полученный результат.}
		{источник постоянного тока, реостат, амперметр, вольтметр.}
		{неидеальный источник тока. КПД и мощность потерь}
	%\item \labtitle
	%	{9}
	%	{Определение мощности человека}
	%	{найти работу, совершаемую при подъеме по лестнице двухэтажного корпуса бегом и шагом. Определить развиваемую при этом мощность.}
	%	{линейка, секундомер, товарищ.}
	%	{механические работа и мощность, закон сохранения энергии.}
	\item \labtitle
		{9, 10}
		{Определение числа $\pi$ методом Монте-Карло}
		{с помощью метода Монте-Карло найти приближённое значение числа $\pi$. \textit{Суть метода Монте-Карло лаборанты расскажут дополнительно.}}
		{линейка, циркуль, несколько мелких тяжелых предметов, копировальная бумага, 2 листа бумаги.}
		{случайные события, вероятность.}
	\item \labtitle
		{9, 10}
		{Измерение длины нити накаливания лампочки}
		{измерить размеры нити накаливания лампочки, используя линзы для переноса изображения.}
		{лампочка с источником питания, собирающая линза, оптический стол.}
		{геометрическая оптика.}
	%\item \labtitle
	%	{9}
	%	{Закон Джоуля-Ленца}
	%	{экспериментально проверить справедливость закона Джоуля-Ленца.}
	%	{вода, нагревательная спираль, регулируемый источник напряжения, термометр, секундомер.}
	%	{электрическая мощность, закон Джоуля-Ленца.}
	\item \labtitle
		{9, 10}
		{Измерение расстояний на местности}
		{измерить на местности расстояние до недоступного объекта.}
		{измерительная лента (рулетка), компас, верёвка, колышки.}
		{метод триангуляции.}
	\item \labtitle
		{9, 10}
		{Исследование мощности лампочки}
		{исследовать зависимость мощности, потребляемой лампой накаливания, от поданного на неё напряжения.}
		{регулируемый источник тока, лампа накаливания, соединительные провода, два вольтметра, известное сопротивление.}
		{закон Ома. Электрическая мощность.}
	\item \labtitle
		{9, 10, 11}
		{Определение фокусного расстояния линз}
		{определить фокусное расстояние собирающей и рассеивающей линз.}
		{лампочка с источником питания, собирающая и рассеивающие линзы, белый лист картона.}
		{геометрическая оптика.}
	\item \labtitle
		{10}
		{Исследование периода колебаний пружинного маятника}
		{Исследовать зависимость периода колебаний пружинного маятника от массы груза. Найти значение периода теоретически, измерив коэффициент упругости отдельно.}
		{Набор грузов известной массы, штатив с лапками для закрепления пружин, линейка, секундомер.}
		{закон Гука. Колебания.}
	\item \labtitle
		{10}
		{Определение коэффициента трения}
		{Определить коэффициент трения между деревом и деревом; между деревом и материалом, покрывающим стол.}
		{Штатив, две ученические линейки.}
		{сухое трение.}
	\item \labtitle
		{10}
		{Определение модуля Юнга}
		{Определить модуль Юнга резинового жгута при деформации растяжения.}
		{Резиновый жгут, миллиметровая бумага, линейка, набор грузов известной массы, штатив.}
		{закон Гука. Модуль Юнга.}
	\item \labtitle
		{10}
		{Определение доли вращательной энергии}
		{Определить долю вращательной энергии при скатывании тела по наклонной плоскости в зависимости от высоты скатывания и угла наклона желоба.}
		{Штатив, желоб, линейка, шарики разных масс, весы, секундомер.}
		{закон сохранения энергии. Вращательное движение.}
	\item \labtitle
		{10}
		{Определение жесткости последовательно и параллельно соединенных пружин}
		{Определить жесткость параллельно и последовательно соединенных пружин экспериментально и теоретически}
		{Две пружины неизвестной жесткости, штатив с лапками, линейка, набор грузов известной массы.}
		{механика. Закон Гука.}
	\item \labtitle
		{10, 11}
		{Маятник Максвелла}
		{Определить момент инерции маятника Максвелла и скорость потерь энергии при его колебаниях.}
		{Штатив, секундомер, линейка, штангенциркуль, маятник Максвелла.}
		{колебания. Вращательное движение.}

	\item \labtitle
		{10, 11}
		{Определение отношения сопротивлений}
		{Определить отношение сопротивлений двух резисторов.}
		{Источник питания, близкий к идеальному, вольтметр, два резистора сопротивлением порядка единиц МОм, соединительные провода.}
		{электричество. Закон Ома.}
	\item \labtitle
		{10, 11}
		{Определение ёмкости конденсатора}
		{Определить емкость конденсатора про помощи баллистического гальванометра. \textit{Суть метода расскажут лаборанты.}}
		{Набор конденсаторов известной емкости, конденсатор неизвестной емкости, микроамперметр, источник напряжения, переключатель.}
		{Электрическая ёмкость, баллистический гальванометр.}
	\item \labtitle
		{10, 11}
		{Определение скорости звука}
		{определить скорость звука по задержке сигнала при распространении в среде}
		{осциллограф, динамик, микрофон, линейка.}
		{колебания, волны.}
	\item \labtitle
		{11}
		{Определение атмосферного давления}
		{Определить атмосферное давление с помощью имеющегося оборудования.}
		{Стеклянная трубка, резиновый шланг, воронка, метр, штатив, линейка.}
		{атмосферное давление. Давление столба жидкости. Газовые законы.}
	\item \labtitle
		{11}
		{Определение площади комнаты}
		{Определить площадь комнаты с помощью математического маятника.}
		{штатив, секундомер, груз, нить.}
		{колебания. Математический маятник.}
	%\item \labtitle
	%	{11}
	%	{Жесткость пружины}
	%	{Экспериментально проверить зависимость периода колебания пружинного маятника от массы груза и жёсткости пружины.}
	%	{штатив, набор грузов, пружины, секундомер, линейка.}
	%	{колебания. Пружинный маятник.}
%	\item \labtitle
%		{11}
%		{Определение ЭДС и внутреннего сопротивления источника}
%		{вычислить ЭДС и внутреннее сопротивление источника тока по результатам измерений силы тока в цепи и напряжения на участке цепи.}
%		{источник постоянного тока, вольтметр, амперметр, два резистора неизвестного сопротивления, соединительные провода.}
%		{закон Ома. Неидеальный источник ЭДС.}
	\item \labtitle
		{11}
		{Вольт-амперная характеристика нелинейного элемента}
		{построить вольт-амперную характеристику нелинейного элемента \textit{на экране осциллографа.}}
		{исследуемый нелинейный элемент, резистор известного сопротивления, источник переменного напряжения, осциллограф.}
		{вольт-амперная характеристика. }
	\item \labtitle
		{11}
		{Измерение постоянной Планка $\hbar$}
		{явление фотоэффекта. Построить зависимость фототока от напряжения на аноде, построить зависимость запирающего напряжения от частоты света. Определить значение постоянной Планка.}
		{источник белого света, набор светофильтров, фотоэлемент, установка для измерения фототока при различных анодных напряжениях}
		{основы квантовой теории света. Фотоэффект}
	\item \labtitle
		{10, 11}
		{Определение доли энергии, рассеиваемой при соударении}
		{определить долю энергии, рассеиваемой шариком при отскоке от различных поверхностей}
		{шарик, микрофон, осциллограф}
		{закон сохранения энергии. Равноускоренное движение.}
	\item \labtitle
		{11}
		{Изучение распространения электромагнитных волн СВЧ диапазона}
		{определить длину волны сверхвысокочастотного (СВЧ) излучателя, определить диэлектрическую проницаемость парафина.}
		{СВЧ-излучатель, осциллограф, активный и пассивные СВЧ-детекторы}
		{электромагнитные волны.}

\end{enumerate}
\end{document}

