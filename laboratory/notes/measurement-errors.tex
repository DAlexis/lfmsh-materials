\documentclass[a4paper,12pt]{article}
\usepackage[T2A]{fontenc}
\usepackage[english,russian]{babel}
\usepackage[utf8]{inputenc} 
\usepackage[vmargin={2cm,2cm},hmargin={2cm,2cm}]{geometry}
\usepackage{graphicx}

\usepackage{amssymb}
\usepackage{amstext}
\usepackage{amsmath}
\usepackage[warn]{mathtext}

\usepackage{indentfirst}
\usepackage{wrapfig}
\usepackage{topcapt}
\usepackage{float}

\title{Расчёт погрешностей}
\newcounter{LFMSHnumber}
\setcounter{LFMSHnumber}{\the \year}
\addtocounter{LFMSHnumber}{-1987}
\date{ЛФМШ-\arabic{LFMSHnumber}, август {\the \year}~г.}


\newcommand{\figRef}[1]{рис.~\ref{#1}}
\newcommand{\FigRef}[1]{Рис.~\ref{#1}}
\newcommand{\Example}{\textbf{Пример: }\par}
\graphicspath{{graph/}}


\begin{document}
%\maketitle
\textit{ЛФМШ-\arabic{LFMSHnumber}, август {\the \year}~г.}
\section{Расчёт погрешностей}

\subsection{Прямые и косвенные измерения}
Измерение той или иной физической величины $A$ называется \textit{прямым,} если её значение непосредственно считывается с измерительного прибора. Например, измерение ширины листа бумаги линейкой, предел измерения которой заведомо больше ширины листа~"--- прямое.

Измерение физической величины $B$ называется \textit{косвенным}, если для её нахождения используются результаты каких-либо других, прямых измерений (существует формула для $B$, в которую входят другие величины). Например, если линейкой измеряется ширина $a$ и длина $b$ бумажного листа, то площадь, полученная как~$S=ab$~"--- косвенно измеренная величина.

\subsection{Абсолютная и относительная погрешности}
Ни одну физическую величину нельзя измерить совершенно точно. \textit{Абсолютной погрешностью} полученного значения~$A$ некоторой физической величины называют такое~$\Delta A$, что с высокой вероятностью 
\begin{equation}
	A-\Delta A < A_{\text{ист}} < A+\Delta A,
\end{equation}
где~$A_{\text{ист}}$~"--- истинное значение измеряемой величины. То есть, истинное значение отличается от полученного не более, чем на $\Delta A$.

Относительной погрешностью этой величины называется 
\begin{equation}
	\varepsilon A = \frac{\Delta A}{A},
\end{equation}
то есть часть, которую составляет $\Delta A$ от $A$. Обычно указывается в процентах.

\subsection{Погрешности прямых измерений}
Абсолютная погрешность $\Delta A$ прямого измерения некоторой величины $A$ в большинстве случаев состоит из двух частей:
\begin{equation}
	\Delta A = \Delta A_{\text{ин}} + \Delta A_{\text{от}},
\end{equation}
Здесь $\Delta A_{\text{ин}}$~"--- или \textit{инструментальная,} или \textit{объективная} погрешность, а $\Delta A_{\text{от}}$~"--- \textit{погрешность отсчёта,} или \textit{субъективная}. Инструментальная погрешность~$\Delta A_{\text{ин}}$ обусловлена неидеальностью измерительного прибора и указывается его изготовителем. Погрешность отсчёта~$\Delta A_{\text{от}}$ характеризует ошибку, возникающую при использовании прибора в конкретной ситуации. Например, при измерении длины стола со скруглёнными краями погрешность отсчёта выше, чем для стола с прямыми. В большинстве случаев погрешность отсчёта не бывает меньше половины цены деления.

\subsection{Погрешности косвенных измерений}

Рассмотрим теперь некоторую величину $C = A + B$, где $A$ и $B$~"--- величины, измеряемые прямо. Пусть $A$ и $B$ имеют абсолютные погрешности $\Delta A$ и $\Delta B$. Тогда:
\begin{equation}
	A-\Delta A < A_{\text{ист}} < A+\Delta A, \quad 
		B-\Delta B < B_{\text{ист}} < B+\Delta B.
\end{equation} 
Складывая выражения, получим:
\begin{equation}
	(A+B)-(\Delta A + \Delta B) < (A_{\text{ист}} + B_{\text{ист}}) < (A+B)+(\Delta A + \Delta B)
\end{equation}
Но, $A+B=C$ и $A_{\text{ист}}+B_{\text{ист}}=C_{\text{ист}}$, тогда
\begin{equation}
	C-(\Delta A + \Delta B) < C_{\text{ист}} < C+(\Delta A + \Delta B),
\end{equation}
а значит, по определению, $\Delta C = \Delta A + \Delta B$, или $\Delta (A+B) = \Delta A + \Delta B$.

Аналогично можно доказать следующие формулы:
\begin{center}
	\begin{tabular}{ cc } 
		\hline
		$\Delta(A\pm B) = \Delta A + \Delta B$ & $\varepsilon (\sqrt{A}) = \frac{1}{2}\varepsilon A $ \\ 
		$\varepsilon (A \cdot B) = \varepsilon A + \varepsilon B$ & $\varepsilon (A^\alpha) = \alpha \varepsilon A $  \\ 
		$\varepsilon \left( \frac{A}{B} \right) = \varepsilon A + \varepsilon B$ & $\varepsilon (e^A) = e^A \varepsilon A $\\ 
		\hline
	\end{tabular}
\end{center}




\end{document}

1